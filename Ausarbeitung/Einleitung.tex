\section{Einleitung}
\begin{itemize}
	\item Parkplatzprobleme allgemein
	\item Alltagsbezug: Parkplatzsuche bei einem Supermarkt (Alle Fahrer haben das gleiche Ziel, Parkplatztopologie ist nicht trivial)
\end{itemize}

\subsection{Zielsetzung}
In dieser Arbeit sollen die folgenden Fragen beantwortet werden
\begin{enumerate}
	\item Wie sollte ein Autofahrer sich verhalten, wenn
	\begin{itemize}
		\item alle 
		\item die meisten
	\end{itemize}
	anderen Autofahrer ihre Parkplätze ohne System wählen?
	\item Welche Strategie setzt sich durch, wenn jeder Autofahrer aktiv eine Strategie verfolgt?
	\item Wie ändern sich die Strategien bei sich ändernder Verkehrslage
	\item Wie lassen sich die vorangegangen Fragen innerhalb des Schulunterrichtes mit Schülern erarbeiten?
\end{enumerate}

\subsection{Vorgehensweise}
\begin{itemize}
	\item Simulation des Problems nach Hutchinson et.al.
	\item Die Strategie und Parameter werden durch einen evolutionären Algorithmus gelernt
	\item Für die 2. Frage werden zudem die Ergebnisse mit Hutchinsons verglichen.
	\item Die Ergebnisse in Form von Ratschlägen für Autofahrer zusammengefasst. 
	\item Abschließend werden die Möglichkeiten für die Behandlung in der Schule mit den Lehrplänen abgeglichen und eine Unterrichtsreihe bzw. ein Projekt geplant.
\end{itemize}