%-----Papier-----%
\documentclass[12pt, a4paper]{article}
\usepackage[a4paper,lmargin={4cm},rmargin={2cm},tmargin={2.5cm},bmargin = {2.5cm}]{geometry}

\usepackage[utf8]{inputenc} %Zeichenunterstützung für Umlaute u.ä.
%-----AMS-----%
\usepackage{amsmath} %Mathepakete
\usepackage{amsfonts}
\usepackage{amssymb}
%-----Layout-----%
\usepackage{ngerman} %Deutsches Layout
\usepackage[onehalfspacing]{setspace} %Zeilenabstand
\usepackage[colorlinks=true,urlcolor=blue, linkcolor=black, citecolor=black]{hyperref} %Links innerhalb des Dokuments, alle schwarz
\sloppy %Blocksatz wird erzwungen (sinnvoll, da Latex Probleme hat, deutsche Wörter zu trennen
%-----Pakete-----%
\usepackage{nameref}
\usepackage{hyperref} % Hyperlinks %http://en.wikibooks.org/wiki/LaTeX/Labels_and_Cross-referencing
\usepackage{graphicx} %Einbinden von Bildern
\usepackage{float} %Erlaubt figures mit option H fest im text zu verankern
\usepackage{caption} %Erlaubt unnummerierte figure-captions
\usepackage{listings} %Erlaubt das (automatische) Einbinden von SourceCode http://en.wikibooks.org/wiki/LaTeX/Source_Code_Listings
\usepackage{tikz} %Zum Zeichnen von Diagrammen http://csweb.ucc.ie/~dongen/LAF/TikZ.pdf
\usepackage{tabularx}%Fuer eigene Tabulararten
\usepackage{longtable}

%-----Literatur-----%
%\usepackage{./din1505/natbib} 
%\usepackage{../../Vorlagen/din1505/natbib} %Dokument liegt eine Ebene unter LaTeX
%\bibliographystyle{./din1505/natdin}
%\bibliographystyle{../../Vorlagen/din1505/alphadin}

\usepackage{pgffor}

%---Deckblatt---%
\newcommand{\myauthor}{Steffen Holzer \& Ellen Wagner}
\newcommand{\mytitle}{}
\newcommand{\subtitle}{}
\newcommand{\dozent}{}
\newcommand{\semester}{}
\newcommand{\logo}{}

\newcounter{Definition}
\newenvironment{definition}[1]{\begin{description}
\stepcounter{Definition}
\item[Definition \theDefinition :]#1\hfill\\}
{\end{description}}

\newenvironment{definition*}[1]{\begin{description}
\item[Definition  :]#1\hfill\\}
{\end{description}}

\newcounter{Satz}
\newenvironment{satz}[1]{\begin{description}
\stepcounter{Satz}
\item[Satz \theSatz :]#1\hfill\\}
{\end{description}}

\newenvironment{satz*}[1]{\begin{description}
\item[Satz :]#1\hfill\\}
{\end{description}}

\newcommand{\degr}{^{\circ}}

%---Tabular---%
\newcolumntype{L}[1]{>{\raggedright\arraybackslash}p{#1}} % linksbündig mit Breitenangabe
\newcolumntype{C}[1]{>{\centering\arraybackslash}p{#1}} % zentriert mit Breitenangabe
\newcolumntype{R}[1]{>{\raggedleft\arraybackslash}p{#1}}

\setcounter{tocdepth}{2}